\documentclass[../main.tex]{subfiles}
\begin{document}
\section*{Technical Acheivements}

Before fully beginning on this project, I have built and tested a rudimentary syntax highlighter and language server. 
\\ \\ 
I have written a syntax highlighter, using TextMate, for a small subset of the Logical English grammar \footnote{The syntax highlighter can be found at https://github.com/nikolaimerritt/LogicalEnglish/tree/main/language-server/logical-english}. This highlighter successfully detected the micro-features \texttt{if} and \texttt{then} as keywords. It also detects the macro-features \texttt{The templates are:} and \texttt{The knowledge base subset includes:} as headers for the template and knowledge base sections, respectively. Once detected, these features were highlighted by the default light and dark theme of Visual Studio Code. 
\\ \\ 
I also followed Jeremy Greer's language server tutorial \cite{blacklist_vscode_tutorial} to create a basic language server. This language server ``blacklists'' certain words by marking their usages as errors. \footnote{The server can be found at https://github.com/nikolaimerritt/LogicalEnglish/tree/main/language-server/blacklist-server. However, the code is entirely taken from the tutorial.} The language server was tested both by manually writing language server requests through standard console input, and by connecting with a Visual Studio Code language client that was taken from the same tutorial. In both tests the server behaved as expected (that is, as described in Greer's tutorial) when ran on Windows Subsystem for Linux. 
\\ \\ 
The purposes of these trial runs were twofold. Firstly, completing these projects allowed us to familiarised ourselves with the tools and languages that I will be working with. Secondly, I have confirmed that a syntax highlighter and language server will run as intended using our choice of platform, tools and testing environment. 
\end{document}