\documentclass[../main.tex]{subfiles}

\begin{document}
\chapter{Project Requirements}
In this section I talk about what LE is, and what my requirements are.
\section{Logical English}
Logical English is a logical and declarative programming language. It is written as a structured document, with a syntax that has few symbols and which closely resembles natural English. \cite{logical_english}.

\subsection{An overview}
Logical English's main goal is to find which literals are true and answer a given question. A literal is a statement, which can be true or false, and cannot be broken down into any smaller statements. Examples include:
\begin{lstlisting}
    fred bloggs eats at cafe bleu.

    emily smith eats at a cafe.

    cafe bleu sells sandwitches.
\end{lstlisting}
Literals may have variables, such as \codeword{a cafe}: these will be discussed further.
\\ 
\\
A Logical English document is chiefly made up of clauses. Clauses are rules that start with a literal and determine when the literal is true. Examples include:
\begin{lstlisting}
    fred bloggs eats at cafe bleu if
        fred bloggs feels hungry.

    emily smith eats at a cafe if
        emily smith feels hungry
        and the cafe sells sandwitches.

    emily smith feels hungry.
\end{lstlisting}
In the second example, \codeword{a cafe} is a variable. This means that if we were later given \codeword{cafe jaune sells sandwitches}, then \codeword{emily smith eats at cafe jaune} would be true.
\\
\\
Templates are used for Logical English to understand which words in a literal correspond to terms (such as \codeword{emily smith} and \codeword{a cafe}), and which words are merely part of the statement (such as \codeword{eats at} or \codeword{feels hungry}). A literal's template is the literal with each of its terms replaced with placeholders. These placeholders start with \codeword{a} or \codeword{an} and are surrounded by asterisks. For example, the literals \codeword{fred bloggs eats at cafe bleu} and \codeword{emily smith eats at a cafe} both share the corresponding template
\begin{lstlisting}
    a person eats at a cafe.
\end{lstlisting} 
In Logical English, each literal needs to have a corresponding a template.

\subsection{The structure of a Logical English program}
Now that literals, clauses and templates have been explained, we can examine a complete Logical English program. An example is provided in Listing \ref{le:short}.
\begin{lstlisting}[caption={A short Logical English program.},label={le:short}]
    the templates are:
    a person travels to a place.
    a place has an amenity.
    
    the knowledge base Travelling includes:
    fred bloggs travels to a holiday resort if 
        the holiday resort has swimming pools.
    
    emily smith travels to a museum if
        the museum has statues 
        and the museum has ancient coins.
    
    scenario A is:
    the blue lagoon has swimming pools.
    the national history museum has statues.
    
    query one is:
    which person travels to which place.
\end{lstlisting}

\subsubsection{Templates}
The program starts with the template section, starting with \codeword{the templates are:}, in which the literals' templates are defined. 

\subsubsection{Knowledge base}
The program's clauses are then given in the knowledge base section. The knowledge base section can either start with \codeword{the knowledge base includes:}, or it can be given a name, in which case it starts with \codeword{the knowledge base <name> includes:}. Clauses are written in order of dependecy: if clause A is referenced by clause B, then clause A must be written before clause B.
\\ 
\\
Clauses begin with exactly one head literal, which is the literal that is logically implied by the rest of the clause. If a clause consists of simply a single head, then the head is taken to be always true. Otherwise, the head literal be followed by an \codeword{if}, then a number of body literals, separated by the connectives \codeword{and}, \codeword{or}, or \codeword{it is not the case that}.
\\ \\
The precedence of these connectives is clarified by indentation: connectives that have higher precedence are indented further. For example, \texttt{(A and B) or C} is written
\begin{lstlisting}
    A
        and B
    or C.
\end{lstlisting}
and \texttt{A and (B or C)} is written
\begin{lstlisting}
    A
    and B
        or C.
\end{lstlisting}
The connective \codeword{it is not the case that} always takes highest precedence. However, there is no default preference over \codeword{and} and \codeword{or}: it is an ambiguity error to write
\begin{lstlisting}
    A 
    and B
    or C.
\end{lstlisting}
In a clause, a variable is introduced for the first time by having its name preceed with \codeword{a} or \codeword{an}. Subsequent uses of the variable must then start with \codeword{the}.

\subsubsection{Scenarios}
Various scenarios can optionally be given. Scenarios contain literals that are used when running a query. Scenarios must have a name, and must start with 
\codeword{scenario <name> is:}. 

\subsubsection{Queries}
The final sections of a Logical English program are the queries. Like scenarios, a query must have a name, and must start with \codeword{query <name> is:}. A question in a query corresponds to a template, with the terms to be found written as placeholders that start with \codeword{which}. 
\\ \\ 
In listing \ref{le:short}, running query one with scenario A yields
\codeword{fred bloggs travels to the blue lagoon.} Query one could also be run with no scenario supplied, but doing so would yield no answer.

\newpage
\section{Project Requirements}
The project will consist of developing two tools for Logical English: a Syntax Highlighter and a Language Server. These two tools will be cross-editor, meaning that they can be used with many of the most popular programming editors with minimal configuration.

\subsection{Syntax Highlighter}
The Syntax Highlighter will identify both micro-features of Logical English such as keywords and variable names, and macro-features such as section headers. It will identify these features using TextMate grammar. This way, the features identified by the grammar can be recognised and styled by the default themes of many popular code editors. 

\subsection{Language Server}
The Language Server will allow the user to generate new templates from rules. If a set of rules do not match any existing templates, the Language Server will communicate this to the editor. It will allow the user to, at the click of a button, generate a template that matches the rules. 
\\ \\ 
If there is time, I will give the language server the feature to alert the user of certain type mismatch errors. The user will be notified of errors where a rule is supplied in the knowledge base with a type that conficts with the corresponding type in the rule's template. To determine whether the one type conficts with the other, the Language Server would consider type inheritance as supported by Logical English.
\\ \\
The language server will communicate with potential language clients using the Langauge Server Protocol. This way, many popular code editors will be able to easily communicate with the language server.

\section{Project Implementation Plan}
The syntax highlighter will be implemented using TextMate grammar, since this has the widest range of editor support.  The Language Server will be implemented in TypeScript using the \texttt{vscode-languageserver} NPM package. This package has clear, thorough documentation which describes multiple example language servers. In testing, both the language server and syntax highlighter will be tested on a Visual Studio Code language client. This choice is made due to Visual Studio Code's powerful debugging features for language plugins.

\subsection{Project Timeline}
The timeline for developing and testing these two tools is below. This plan has us completing both the template generation and type error detection features of the language server. However, if any large problems arise, I will prioritise solving these over working on type error detection.
\\
\begin{tabularx}{\textwidth}{|l|X|}
    \hline
    6th June - 10th June & 
    Write a TextMate grammar for Logical English.
    \\
    \hline
    13th June - 17th June & 
    Using the Visual Studio Code documentation \cite[]{vsc_langserver_features}, create a proof-of-concept language server with dummy error highlighting, warning highlighting, and code generation.
    \\ \hline
    20th June - 25th June & 
    In the language server, convert Logical English templates to a suitable TypeScript representation. \newline
    Using this representation, determine whether a Logical English rule conforms to a template.
    \\ \hline
    27th June - 8th July & 
    Create a template from first two, then arbitrarily many, rules.
    \\ \hline
    11th July - 23rd July & 
    Create a TypeScript representation for Logical English types, to be used by the language server. \newline
    Use this type representation in to augment the template representation with types of the template's variables. \newline
    Consider types when determining whether a rule conforms to a template.
    \\ \hline
\end{tabularx}
\end{document}
