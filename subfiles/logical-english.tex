\documentclass[../main.tex]{subfiles}

\begin{document}
\chapter{Logical English}
\label{chapter:le}
% This chapter introduces the features of Logical English, then relates them to the functional requirements of the editor.
% \todo[inline]{Use screenshots of the editor.}
% \todo[inline]{Rewrite examples to use beverages rather than `cafe X'.}
% \section{Logical English}
Logical English is a logical and declarative programming language. Logical English is an example of a controlled natural language; it is written as a structured document, with a syntax that has few symbols and which resembles natural English. \cite{logical_english}.

\section{A brief overview}
Logical English allows knowledge bases to be represented as logical rules. These logical rules can then be used to answer queries. 

\subsection{Atomic Formulas}
A fundamental concept in Logical English is the atomic formula. An atomic formula is a statement which cannot be broken down into any smaller statements. Examples in Logical English include:
\begin{lstlisting}[language={LE},caption={An example of three atomic formulas in Logical English.},label={le:atomic formulas}]
    fred bloggs eats at cafe bleu.
    disha khan eats at cafe jaune.
    cafe bleu sells sandwiches.
\end{lstlisting}
% As in listing \ref{le:atomic formulas}, atomic formulas may include constants, such as \codeword{fred bloggs} and \codeword{cafe bleu}, or variables, such as \codeword{a cafe}. These will be discussed further.

\subsection{Clauses}
\label{section:le-clauses-one}
Clauses are rules that determine when an atomic formula can be logically derived. Examples include:
\newpage
\begin{lstlisting}[language={LE},label={le:clauses},caption={Examples of clauses in Logical English.}]
    fred bloggs eats at cafe bleu if
        fred bloggs feels hungry.

    disha khan eats at a cafe if
        disha khan feels hungry
        and the cafe sells sandwiches.

    disha khan feels hungry.
\end{lstlisting}
Clauses begin with a `conclusion', which is the atomic formula that is logically implied by the rest of the clause. If there are no other atomic formulas, then the conclusion is logically implied in all cases. Otherwise, the conclusion is followed by the keyword \codeword{if}, then a number of atomic formulas that logically imply the conclusion. These atomic formulas are separated by the connectives \codeword{and}, \codeword{or}, or \codeword{it is not the case that}.
\\
\\
A variable is introduced for the first time in a clause by beginning its name with \codeword{a} or \codeword{an}. In Listing \ref{le:clauses}, \codeword{a cafe} is a variable. Subsequent uses of the same variable start with \codeword{the}. This means that in Listing \ref{le:clauses}, if we were later given (or could later derive) that
\begin{lstlisting}[language={LE}]
    cafe jaune sells sandwiches
\end{lstlisting}
then the conclusion \codeword{disha khan eats at cafe jaune} would be logically implied through the second clause. 

\subsection{Templates}
The words in an atomic formula can have one of two functions. A word could either be part of a \textit{term} of the atomic formula, which is an object that the atomic formula describes. Otherwise, a word is part of the atomic formula's \textit{predicate}: the name of the assertion that is applied to the terms. Viewed this way, terms are also known as \textit{predicate arguments}.
\\
\\
Templates are used in Logical English to clarify which parts of the atomic formula are terms, and which words are part of the predicate. An atomic formula's template is written as the atomic formula with each of its terms replaced with placeholders. These placeholders start with \codeword{a} or \codeword{an}, and are surrounded by asterisks. 
\\
\\
For example, the template
\begin{lstlisting}[language={LE},caption={A template in Logical English},label={le:template}]
    *a cafe* serves *an item* from *a time* to *a time*.
\end{lstlisting}
is a template for the atomic formula
\begin{lstlisting}[language={LE}]
    cafe jaune sells crepes from breakfast to noon.
\end{lstlisting}
This template allows the terms to be identified as \codeword{cafe jaune}, \codeword{crepes}, \codeword{breakfast} and \codeword{noon}. In Logical English, every atomic formula must have a corresponding a template.

\subsection{Higher-order Atomic Formulas}
A higher-order atomic formula contains another atomic formula as a term. A higher-order atomic formulas can be written with the keyword \codeword{that} preceding the atomic formula that it describes. An example is given in Listing \ref{le:meta}.
\begin{lstlisting}[language=LE,label={le:meta},caption={An example of an atomic formula featuring in a meta atomic formula.}]
the templates are:
*a bank* receives money.
*a person* pays money to *a bank*.
there is a requirement that *an action*.

the knowledge base Payments includes:
LE bank receives money if
    there is a requirement that bob pays money to LE bank.
\end{lstlisting}
In Listing \ref{le:meta}, the atomic formula \codeword{bob pays money to LE bank} features in the higher-order atomic formula \codeword{there is a requirement that __}.

\section{Program Structure}
A complete Logical English program features atomic formulas, clauses and templates. An example is provided in Listing \ref{le:short}.
\newpage
\begin{lstlisting}[language={LE},caption={A short Logical English program.},label={le:short}]
the templates are:
*a person* travels to *a place*.
*a place* has *an item*.

the knowledge base Travelling includes:
fred bloggs travels to a holiday resort if 
    the holiday resort has swimming pools.

disha khan travels to a museum if
    the museum has statues 
    and the museum has ancient coins.

scenario A is:
the blue lagoon has swimming pools.
the national history museum has statues.

query one is:
which person travels to which place.
\end{lstlisting}
The rest of this section expands on the features shown in Listing \ref{le:short}, as well as other features of Logical English.
\subsection{Templates}
The program starts with the template section, with the section title \codeword{the templates are:}, in which the templates are defined. 
\\
\\
Each Logical English  document is also implicitly given several general-purpose templates: templates such as \codeword{*a thing* = *a thing*}, or \codeword{*a date* is immediately before *a date*}. A full list of these templates can be found in the Logical English handbook \cite{le_handbook}.

\subsection{Clauses}
\label{section:knowledge-base}
\subsubsection{The Knowledge Base}
The program's clauses are given in the knowledge base. The knowledge base starts with the section title \codeword{the knowledge base <name> includes:}, where \codeword{<name>} is a name that can optionally be given.
% \todo[inline]{What does this name do?}

\subsubsection{Connectives in Clauses}
The atomic formulas in a clause are separated by logical connectives. The precedence of these connectives is determined by indentation; connectives that have higher precedence are indented further. For example, what one might write in English as \texttt{(A and B) or C} is written
\begin{lstlisting}[language={LE}]
    A
        and B
    or C.
\end{lstlisting}
and \texttt{A and (B or C)} is written
\begin{lstlisting}[language={LE}]
    A
    and B
        or C.
\end{lstlisting}
There is no default preference over \codeword{and} and \codeword{or}: it is an ambiguity error to write
\begin{lstlisting}[language={LE}]
    A 
    and B
    or C.
\end{lstlisting}

\subsubsection{Categories of terms}
\label{section:categories-of-terms}
Logical English supports three categories of terms:
\begin{enumerate}
    \item \textbf{Constants}. These terms represent a single value and cannot be broken down into any constituent terms. Examples include names such as \codeword{mary jane} or \codeword{LE Bank}, as well as numbers.
    \item \textbf{Complex terms}. These are terms that are built from other terms. Examples include lists, such as \codeword{[one, two, three]}, that are built from any other terms, as well as dates, such as \codeword{01-01-1970}, which are built from numbers. An atomic formula can also be a complex term, since atomic formulas can include other terms, and can feature as arguments of higher-order atomic formulas.
    \item \textbf{Atomic Formulas}. Atomic formulas are terms, since they can feature as arguments of higher-order atomic formulas. Similar to complex terms, atomic formulas can contain other terms. However, unlike complex terms, atomic formulas have a logical meaning: clauses are made up of atomic terms.
    \item \textbf{Variables}. Variables are terms that range over other terms. The syntax of variables is described in Section \ref{section:le-clauses-one}. 
\end{enumerate}
% Later on in this report I will refer to terms that are not atomic formulas as \textit{data}. 
% \todo[inline]{These names are based on the terminology of Prolog, \url{https://www.let.rug.nl/bos/lpn//lpnpage.php?pagetype=html&pageid=lpn-htmlse2}}

\subsection{Scenarios}
Various scenarios can optionally be given. Scenarios contain clauses that are just conclusions, and can be used when running a query. Scenarios must have a name, and begin with the section title
\codeword{scenario <name> is:}. 

\subsection{Queries}
The final section of a Logical English program contains the queries. A query consists of one or more questions that the Logical English engine seeks to answer. These questions ask for which atomic formulas can be deduced by the Logical English engine. The terms that the query is to look for are written as placeholders that start with \codeword{which}. 
\\
\\
For example, running query one with scenario A from \ref{le:full}
\begin{lstlisting}[language={LE},label={le:full},caption={Another short Logical English program.}]
    the templates are:
    *a person* visits *a resort*.
    *a resort* has *an amenity*.

    the knowledge base Resorts includes:
    fred bloggs visits a resort if
        the resort has swimming pools.

    scenario A is:
    the blue lagoon has swimming pools.

    query one is:
    which person travels to which resort.
\end{lstlisting}
gives the answer
\begin{lstlisting}[language={LE}]
    fred bloggs travels to the blue lagoon.
\end{lstlisting}
Query one could also be run with no scenario supplied, but doing so would yield no answer.

\end{document}