\documentclass[../main.tex]{subfiles}

\begin{document}
\chapter{Introduction}
\section{Abstract}
Language Extensions for code editors are a crucial tool in writing code quickly and without errors. In this project, I create a language extension for the logical, declarative programming language Logical English. The language extension highlights the syntactic and semantic features of Logical English, identifies errors, and generates ``boilerplate'' code to fix them. The language extension uses the Language Server Protocol and is therefore cross-editor. It is evaluated when connected to the Visual Studio Code and Codemirror IDEs.

\section{Motivation: why a new editor?}
Logical English is a relatively new programming language, first introduced in late 2020 \cite{logical_english} \todo{is this where LE was first introduced?}. Although Logical English has an online editor hosted on the SWISH platform \cite{swish_editor}, the editor is not user-friendly. SWISH is primarily a Prolog editor, and so any Logical English code has to be written in a long string that is, in the same file, passed to a Prolog function to be interpreted. 
\\ 
\\ 
This has significant drawbacks: since Logical English code is written in a Prolog string, the Logical English content cannot be treated by the editor as a standalone program. This means that it can receive no syntax highlighting, error detection, or code completion features beyond that of a Prolog string. Since these features are essential for productivity \todo[inline]{find a source here about IDEs vs Notepad}, a new editor was needed that was custom-built for writing Logical English.

\end{document}