\documentclass[../main.tex]{subfiles}

\begin{document}
\chapter{Introduction}
\section{Abstract}
Language Extensions for code editors are a crucial tool in writing code quickly and without errors. In this project, I create a language extension for the logical, declarative programming language Logical English. The language extension highlights the syntactic and semantic features of Logical English, predicts the completion of atomic formulas, identifies errors and generates ``boilerplate'' code to fix them. The language editor also extends the language of Logical English through type checking the use of terms by introducing a hierarchical type system. The extension is based on a language server that uses the Language Server Protocol. It is evaluated when connected to a Visual Studio Code front-end.

\section{Structure of the Document}
The first two chapters of this document introduce Logical English and the Logical English editor's features. These chapters are intended for prospective users who are unfamiliar with either. These two chapters also set the scene for the remainder of the document: due to the complexity of the editor, and the language for which it is built, it may help to begin with a birds-eye view of the final product. In the remainder of the document, the project of building the Logical English editor is discussed.  These chapters describe the requirements, research, design and implementation of the editor, evaluate the finished product, and lay out how this editor is connected to the wider body of surrounding literature.

\section{Motivation: why a new editor?}
Logical English is a relatively new programming language, first introduced under that name in late 2020 \cite{logical_english}. Although Logical English has an online editor hosted on the SWISH platform \cite{swish_editor}, the editor lacks features to edit Logical English that are common to most programming language editors. The SWISH editor is primarily a Prolog editor, so any Logical English code has to be written in a string that is an argument to a Prolog function. Since Logical English code is written in a Prolog string, the Logical English content cannot be treated by the editor as a standalone program. This means that it can receive no syntax highlighting, error detection, or code completion features beyond that of a Prolog string. Since these features are essential for productivity \cite{ide_productivity}, a new editor was needed that was custom-built for writing Logical English.
% \todo[inline]{Find a source about the productivity of IDEs vs a plaintext editor}
\end{document}